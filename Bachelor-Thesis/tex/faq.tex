%# -*- coding: utf-8-unix -*-
% !TEX program = xelatex
% !TEX root = ../thesis.tex
% !TEX encoding = UTF-8 Unicode
\chapter{常见问题}
\label{chap:faq}

{\bfseries{}Q:我是否能够自由使用这份模板?}

A:这份模板以Apache License 2.0开源许可证发布,请遵循许可证规范。

{\bfseries{}Q:我的论文是Word排版的,学校图书馆是不是只收 \LaTeX 排版的论文?}

A:当然不是,Word版论文肯定收。

{\bfseries{}Q:我的论文是 \LaTeX 排版的,学校图书馆是不是只收Word排版的论文?}

A:当然不是,PDF版的电子论文是可以上交的。是否要交Word版就看你导师的喜好了。

{\bfseries{}Q:为什么屏幕上显示的左右页边距不一样?}

A:模板默认是双面打印,迎面页和背面页的页边距是要交换的,多出来的那一部分是留作装订的。

{\bfseries{}Q:为什么在参考文献中会有“//”符号?}

A:那就是国标GBT7714参考文献风格规定的。但可以使用 gbpunctin=false 选项将其还原成 in:,进一步可以在导言区加入\verb+\DefineBibliographyStrings{english}{in={}}+将其去掉。

{\bfseries{}Q:为什么参考文献中会有[s.n.],[S.l], [EB/OL]等符号?}

A: 那也是国标GBT7714参考文献风格定义的。[s.n.]表示出版者不祥,[S.l]表示出版地不祥,[EB/OL]表示引用的参考文献类型为在线电子文档。但可以使用gbpub=false 选项将其缺省补充的出版项[s.n.]等去掉。也可以使用选项 gbtype=false 将参考文献类型标识去掉。

{\bfseries{}Q:如何获得帮助和反馈意见?}

A:你可以通过\href{https://github.com/sjtug/SJTUThesis/issues}{在github上开issue}
、在\href{https://bbs.sjtu.edu.cn/bbsdoc?board=TeX_LaTeX}{水源LaTeX版}发帖反映你使用过程中遇到的问题。

{\bfseries{}Q:使用文本编辑器查看tex文件时遇到乱码?}

A:请确保你的文本编辑器使用UTF-8编码打开了tex源文件。

{\bfseries{}Q:在CTeX编译模板遇到“rsfs10.tfm already exists”的错误提示?}

A:请删除\verb+X:\CTEX\UserData\fonts\tfm\public\rsfs+下的文件再重新编译。问题讨论见\href{https://bbs.sjtu.edu.cn/bbstcon,board,TeX_LaTeX,reid,1352982719.html}{水源2023号帖}。

{\bfseries{}Q:升级了TeXLive 2012,编译后的文档出现“minus”等字样?}

A:这是xltxtra和fontspec宏包导致的问题。学位论文模板从0.5起使用metatlog宏包代替xltxtra生成 \XeTeX 标志,解决了这个问题。

{\bfseries{}Q:为什么在bib中加入的参考文献,没有在参考文献列表中出现?}

A: bib中的参考文献条目,常通过\verb+\cite+或\verb+\parencite+或\verb+\supercite+或\verb+\textcite+等命令在正文中引用进而加入到参考文献列表中。当需要将参考文献条目加入到文献表中但又不引用可以使用\verb+\nocite+命令,当nocite参数为*时则引入bib中的所有文献。
%\verb+\upcite+ 是哪个宏包的?之前没有见过

{\bfseries{}Q:我可以使用Sublime Text编写学位论文吗?}

A: 可以。首先\href{https://www.sublimetext.com/}{下载}并安装Sublime Text,然后安装
\href{https://packagecontrol.io/installation}{Package Control},
之后按\verb|ctrl+shift+p|或者\verb|cmd+shift+p|调出命令窗口,
输入\verb|install|,选择\textit{Package Control: Install Package},按回车,
稍等片刻,等待索引载入后会弹出选项框,输入\verb|LaTeXTools|并回车,即可成功安装插件。
之后只需要打开\verb|.tex|文件,按\verb|ctrl+b|或者\verb|cmd+b|即可编译,
如有错误,双击错误信息可以跳转到出错的行。

{\bfseries{}Q:在macTex中,为什么pdf图片无法插入?}

A:如果报错是“pdf: image inclusion failed for "./figure/chap2/sjtulogo.pdf".”,则采取以下步骤

\begin{lstlisting}[basicstyle=\small\ttfamily, caption={编译模板}, numbers=none]
brew install xpdf
wget http://mirrors.ctan.org/support/epstopdf.zip
unzip epstopdf.zip
cp epstopdf/epstopdf.pl /usr/local/bin/
cd figure/chap2
pdftops sjtulogo.pdf
epstopdf sjtulogo.ps
pdfcrop sjtulogo.pdf
mv sjtulogo.pdf backup.pdf
mv sjtulogo-crop.pdf sjtulogo.pdf
\end{lstlisting}

{\bfseries{}Q:为什么维普等查重系统无法识别此模板生成的 pdf 内所有的中文?}

A: 中文无法识别的情况多半是由于使用了 ShareLaTeX 的原因,请尝试使用 TexStudio 等软件在本地进行编译。
如果使用 TeXstudio 请在 Preferences-Build 中将 Default Compiler 和 Default Bibliography Tool 分别改为 XeLaTeX 和 Biber。

{\bfseries{}Q:如何向你致谢?}

A: 烦请在模板的\href{https://github.com/sjtug/SJTUThesis}{github主页}点击“Star”,我想粗略统计一下使用学位论文模板的人数,谢谢大家。非常欢迎大家向项目贡献代码。
